%B5サイズの指定
\documentclass[b5paper, titlepage, 11pt]{jsarticle}
\usepackage[utf8]{inputenc}

%余白の設定 http://joker.hatenablog.com/entry/2012/07/09/153537
\usepackage[top=25truemm,bottom=25truemm,left=34truemm,right=25truemm]{geometry} 

%図を使います
\usepackage[dvipdfmx]{graphicx}

%文献スタイルの設定
\usepackage{kyotosoc_bib}

%注のスタイルの設定 http://www.latex-cmd.com/struct/footnote.html
\renewcommand{\thefootnote}{(\arabic{footnote})}

%文字数と行数の指定 https://rion778.hatenablog.com/entry/20091002/1254482262
\makeatletter
\def\mojiparline#1{
    \newcounter{mpl}
    \setcounter{mpl}{#1}
    \@tempdima=\linewidth
    \advance\@tempdima by-\value{mpl}zw
    \addtocounter{mpl}{-1}
    \divide\@tempdima by \value{mpl}
    \advance\kanjiskip by\@tempdima
    \advance\parindent by\@tempdima
}
\makeatother
\def\linesparpage#1{
    \baselineskip=\textheight
    \divide\baselineskip by #1
}

%URL表示のため
\usepackage{url}

%回帰分析の結果表で変数間を区切るための点線の設定
%いらなかったら消してください
\usepackage{arydshln}
\setlength{\dashlinedash}{0.75pt}
\setlength{\dashlinegap}{1.5pt}


\title{社会学用LaTexスタイルファイルkyotosoc}
\author{佐藤 慧}
\date{\today}

%-----------------documentはじまり---------------------------------
% タイトル
\begin{document}
\maketitle

% 目次ページ
%目次は\setcounter{tocdepth}{2}で深さを2にできる.
\setcounter{tocdepth}{2}
\tableofcontents

% 一行あたり文字数の指定 https://rion778.hatenablog.com/entry/20091002/1254482262
\mojiparline{32}
% 1ページあたり行数の指定
\linesparpage{25}

%------------------本文はじまり--------------------------------------
\section{はじめに}
京大文学部社会学の卒論で使ったLaTexの設定を公開します.Bib関係は樋口耕一先生の\textbf{nissya.bst}と\textbf{nissya\_bib.sty}\footnote{\url{http://koichi.nihon.to/psnl/latex0.html}}を下敷きに\textbf{kyotosoc.bst}と\textbf{kyotosoc\_bib.sty}をつくり,版組はtexファイルのプリアンブルにいろいろと書き込んで指定しました.\textbf{nissya\_bib}は大変有用ですが,環境の違いなどによってうまくいかないことがありますので,スタイルファイルのバージョンを増やすことも大事かと思い,配布することにしました.少しスクリプトを書き換えるだけで様々なフォーマットにできます.


% むしろ主目的は\textbf{nissya.bst}と\textbf{nissya\_bib.sty}の簡単な書き換え方を示すことにあります.\textbf{nissya_bib}は社会学界隈でおそらく唯一のBibTexスタイルファイルとして大変有用ですが,環境の違いなどによってうまくいかないことがあります.

\section{使用環境}
私はTex環境をデスクトップにはダウンロードせずに,Overleaf\footnote{\url{https://ja.overleaf.com}}を使いました.OSはmacOs Catalina 10.15.1で,ブラウザはGoogle Chromeです.

\section{どのように改変したか}
それぞれの改変箇所については,ファイル本体の冒頭に直接書き込んでありますが,

\begin{enumerate}
    \item チルダを半角スペースに置き換え
    \item ¥マークを\verb|\|に置き換え
\end{enumerate}

この文字コード関係の改変が主なものです.結局のところ文字コードで引っかかる場合が多いと思われます.それ以外は京大のローカルルールに合わせた書式の改変ということになります.むしろ,texファイル本体のプリアンブルで制御する部分が多いです.\textbf{sample.tex}のプリアンブルをご参照ください.改変すべき箇所と,対応するファイルの関係は以下のようなようなものです.

\begin{itemize}
    \item 余白:プリアンブルで
    \item 文字数と行数:プリアンブルで
    \item 本文中の文献参照と参考文献リスト:\textbf{kyotosoc.bst}で
        \begin{itemize}
            \item ただし,パンクチュエーションは\textbf{kyotosoc\_bib.sty}の最後の箇所で
        \end{itemize}
    \item 注:プリアンブルで
\end{itemize}

\section{例}
本文中での参照は,\citet{平山洋介2018富か}のような場合は\verb|\citet{当該本のラベル}|,丸括弧内で参照する場合\citep[50]{平山2009}は\verb|\citep[50]{当該本のラベル}|とします.詳しくは前掲の樋口先生のサイトでお確かめください.

\subsection{さらに例}

\begin{itemize}
    \item 持ち家の所有はライフチャンスを大きく左右する\citep{zavisca_socioeconomic_2016}.
    \item \citet{schwartz2009}によれば,政府の住宅政策の方針は国によって大きく異なる.
\end{itemize}

%------------------本文おわり-------------------------------------
% \clearpage
%文献リストの指定
\bibliography{sample}
%文献スタイルの指定
\bibliographystyle{kyotosoc}

\end{document}
%------------------documentおわり-------------------------------------
